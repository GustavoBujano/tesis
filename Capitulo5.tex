\section{Conclusión}

En conclusión, podemos mencionar que aumentar la seguridad y confort en un vehículo de gama baja es un trabajo con grandes intereses. El conocer el nivel del óxido de nitrógeno que emite el vehículo también es un parámetro interesante para enviar el automóvil a revisión y contribuir al medio ambiente\\

Este proyecto se puede llevar a cabo sin problemas con vehículos de años no recientes, actualmente existen carros nuevos que no tienen ciertos actuadores que se muestran en este proyecto. Sin embargo, no se recomienda instarlos dentro de un vehículo en un tiempo menor a cinco años del modelo del carro debido a que puede tener problemas en el tema de garantía. Las agencias automotrices permiten el mantenimiento preventivo y correctivo los primeros cinco años, después no desean dar mantenimiento a los vehículos debido al tiempo del auto.\\

Durante la elaboración del proyecto se ha llegado a comprender de mejor manera el funcionamiento del vehículo, en específico la parte eléctrica o electrónica, según sea el caso.\\

Durante el trabajo previo se llevo la tarea de provar la funcionalidad del Scanner EML327 y se llego a la conclusión que, aunque muestra información sobre la velocidad, RPM, entre otros. No permite la manipulación de la información de los sensores o de los actuadores y los datos arrojados son pertenecientes al sistema de tracción.\\

Las herramientas y accesorios que se utilizaron para este proyecto son fáciles de encontrar y no se necesitan lugares especializados para comprar. \\
En la mayoría de las actividades que se probaron es conveniente su funcionamiento, a excepción del encendido automático debido a factores que pueden influir negativamente en la seguridad y confort del vehículo, como es el caso de que la marcha se llegue a quemar o que la pila sea descargada por dejar el modo de ignición encendido.\\

Un apartado muy importante es el historial de la aplicación, debido a que registra el patrón de servicios que utiliza el usuario en la aplicación, con una muestra de datos convicente se pueden crear nuevos comandos automatizados.\\

Algunos casos que se proponen son:\\
\begin{enumerate}
\item Al apagar el vehìculo desactivar los seguros.
\item Al activar los seguros subir los vidrios.
\item Al desactivar los seguros bajar los vidrios.
\item Encender las luces exteriores a cierta hora si el vehículo esta encendido.
\end{enumerate}

Cabe mencionar que algunos de los casos que se proponen existen en vehículos modernos, sin embargo, se deben instalar de acuerdo al usuario. ya que existen usuarios por ejemplo, que no bajan los vidrios debido a que encienden el aire acondicionado, entre otros ejemplos.\\

Como trabajo a futuro se puede analizar los datos y proponer nuevos casos de automtización, además de mejorar la aplicación y agregarle nuevas funcionalidades, también es recomendable desarrollar un sistema de comunicación bidireccional entre los actuadores, de tal forma que el usuario pueda estar enterado en caso de un error con algun actuador.

