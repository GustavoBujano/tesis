Las tecnologías de la información son parte fundamental de los grandes cambios que se están realizando, desde automatización de procesos hasta la creación de nuevos sistemas de información. La industria de los automóviles está enfocada a crear vehículos que cumplan una mejor experiencia para los tripulantes a la hora de desplazarse de un lugar a otro, creando vehículos con mayor comodidad y seguridad. Sin embargo, estas industrias están trabajando con tecnología de punta que no está accesible al público en general, además de que existen muy pocos procesos para poder incluir tecnología accesible a vehículos de modelos menos recientes.\\[\separacionCorta]

El presente trabajo propone una metodología para incluir tecnología accesible a vehículos de gama baja, con la intención de aumentar la satisfacción del tripulante a la hora de viajar. La implementación de actuadores permite agregar seguridad y confort al vehículo, así como brindarle una seguridad al dueño del mismo.\\[\separacionCorta]

La metodología propone implementar en un vehículo GOLF 91 GL actuadores para obtener una mejor experiencia a un costo bajo, comparado con obtener un vehículo que tenga estas herramientas de mejora.\\[\separacionCorta]

Los resultados son la automatización de algunos parámetros mediante la creación de una aplicación móvil que manipula ciertos actuadores incorporados en el vehículo a través de tecnología bluetooth como medio de comunicación y concurrentemente monitorea la emisión de algunos gases del vehículo.\\[\separacionCorta]

En conclusión, se puede mencionar que la integración de tecnología al vehículo GOLF, ha sido satisfactoria, debido a que existe una manipulación de ciertos actuadores a través de la aplicación móvil desarrollada, la implementación de un monitoreo continuo de la emisión de algunos gases ayuda a verificar el estado del vehículo y notificar una irregularidad de contaminación ambiental.\\[\separacionCorta]
