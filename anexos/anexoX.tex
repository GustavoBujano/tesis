\section{Anexo 1}
\subsection{Codigo para obtener los datos del Óxido de Nitrógeno}
Obtención de los datos en partes por millón (ppm) del sensor de gas óxido de nitrógeno.

\begin{verbatim}

//define la entrada analogica para el sensor
#define MQ1 (0) 
//define el valor de la resistencia mde carga en kilo ohms
#define RL_VALOR (5)     
// resistencia del sensor en el aire limpio / RO,
// que se deriva de  la tabla de la hoja de datos
#define RAL (9.83) 
#define GAS_LP  (0)
//Cadena recibida desde el PC
String inputstring = "";                                                        
float           LPCurve[3]  =  {2.3,0.21,-0.47};
float           Ro           =  10;
void setup(){
//Inicializa Serial a 9600 baudios
Serial.begin(9600);                                                                  
 Serial.println("Iniciando ...");
   //configuracion del sensor
  Serial.print("Calibrando...\n");
  //Calibrando el sensor. Por favor de asegurarse que el sensor se encuentre
  // en una zona de aire limpio mientras se calibra
  Ro = Calibracion(MQ1);          
  Serial.print("Calibracion finalizada...\n");
  Serial.print("Ro=");
  Serial.print(Ro);
  Serial.print("kohm");
  Serial.print("\n");
}
 
void loop()
{
   Serial.print("LP:");
   Serial.print(porcentaje_gas(lecturaMQ(MQ1)/Ro,GAS_LP) );
   Serial.print( "ppm" );
   Serial.print("    ");
   Serial.print("\n");
   delay(200);
}
 
float calc_res(int raw_adc)
{
  return ( ((float)RL_VALOR*(1023-raw_adc)/raw_adc));
}
 
float Calibracion(float mq_pin){
  int i;
  float val=0;
    //tomar múltiples muestras
    for (i=0;i<50;i++) {                                                                               
    val += calc_res(analogRead(mq_pin));
    delay(500);
  }
  //calcular el valor medio
  val = val/50;                                                                                         
  val = val/RAL;
  return val;
}
 
float lecturaMQ(int mq_pin){
  int i;
  float rs=0;
  for (i=0;i<5;i++) {
    rs += calc_res(analogRead(mq_pin));
    delay(50);
  }
rs = rs/5;
return rs;
}
 
int porcentaje_gas(float rs_ro_ratio, int gas_id){
   if ( gas_id == GAS_LP ) {
     return porcentaje_gas(rs_ro_ratio,LPCurve);
   }
  return 0;
}
 
int porcentaje_gas(float rs_ro_ratio, float *pcurve){
  return (pow(10, (((log(rs_ro_ratio)-pcurve[1])/pcurve[2]) + pcurve[0])));
}
\end{verbatim}
